\documentclass{article}

\usepackage{amsfonts}
\usepackage{amsmath}
\usepackage{amsthm}
\usepackage[margin=1in]{geometry}
\usepackage{hyperref}

\title{\href{https://math.umn.edu/sites/math.umn.edu/files/exams/complexf18.pdf}{Fall 2018 Complex Analysis Preliminary Exam}}
\author{University of Minnesota}
\date{}
\begin{document}
\maketitle

Where possible, computations have been also done using SageMath code available on GitHub at \\ github.com/tekaysquared/prelims (feel free to make pull requests!)

\begin{enumerate}


	\setcounter{enumi}{1}
	
	\item Write the Laurent expansion of $f(z) = \frac{1}{z^4-1}$ centered at $0$ and convergent in $|z|>1$.
	
	\begin{proof}
		Factor out $z^{-4}$ to see that 
		\begin{align*}
			f(z) &= \frac{1}{z^4 (1 - {1}/{z^4})}\\
			&= \frac{1}{z^4} \sum_{n=0}^\infty \frac{1}{z^{4n}} \\
%			&= \frac{1}{z^4} \left ( \right )
		\end{align*}
		which converges for $|1/z^4| <1$ which is to say for $|z|>1$.
		So then $f$ has a Laurent expansion
		\[f(z) = \sum_{n = - \infty}^{\infty} a_n z^{n} \]
		where 
		\[ a_n = \begin{cases}
						  1 & n = -4k \text{ for nonzero positive integers }k\\
						  0 & \text{otherwise}
		\end{cases}\]
	\end{proof} 
	
	\setcounter{enumi}{5}
	\item Determine the radius of convergence of the power series for $\log z$ at $z_0 = -4 + 3i$.
	
	\begin{proof}
		Let $R$ denote the radius of convergence of the power series of $\log z$ centered at $z_0$.
		Now, note that there is no logarithm which takes a value at $0$, since $e^w = 0$ is never true for $w \in \mathbb{C}$.
		Thus, the power series expansion can converge for a disk of radius at most $|-4+3i - 0| = 5$, and so 
		\[ R \leq 5.\]
		
		On the other hand, it is a theorem that if $\Omega$ is a simply connected subset of 
		$\mathbb{C}$ which does not contain $0$ then there is a branch of the logarithm which is holomorphic on $\Omega$. Observe that the open disk $D_{5}(-4+3i) := \{ z \in \mathbb{C} : |-4+3i - z | < 5\}$ is simply connected and does not contain zero. Thus, there is a branch of the logarithm (call it $\log_{D_5}$) which is holomorphic on $D_5$. Since we have constructed a disk of radius 5 on which there is a holomorphic logarithm, we see that 
		\[ R \geq 5.\]
		Since we have bounded $R$ both above and below by 5, we see that $R=5$.		
	\end{proof}
	
\end{enumerate}


\end{document}