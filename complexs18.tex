\documentclass{article}

\usepackage{amsthm}
\usepackage{amsmath}
\usepackage{amsfonts}
\usepackage[margin=1in]{geometry}
\usepackage{hyperref}

\DeclareMathOperator{\res}{res}

\title{\href{https://math.umn.edu/sites/math.umn.edu/files/exams/complexs18.pdf}{Spring 2018 Complex Analysis Preliminary Exam}}
\author{University of Minnesota}
\date{}
\begin{document}
\maketitle

Where possible, computations have been also done using SageMath code available on GitHub at \\ github.com/tekaysquared/prelims (feel free to make pull requests!)

\begin{enumerate}
	
	\item Write the first three terms of the Laurent expansion of $\displaystyle f(z) = \frac{1}{z(z-1)(z-2)}$ centered at $0$ and convergent in $|1| < z < |2|$
	
	\begin{proof}
		The core idea of the computation is to split the function into a product of power series.
		First, we observe that 
		\[ \frac{1}{z-1} = \frac{1}{z(1-1/z)}\] 
		and see the geometric series 
		\[ \frac{1}{1-1/z} = \sum_{n=0}^\infty \left( \frac{1}{z} \right)^n,\]
		which converges for $|1/z| < 1$, or equivalently $|z|>1$.
		Similarly we see that 
		\[ \frac{1}{z-2} = \frac{-1}{2(1-z/2)} = -\frac{1}{2} \sum_{n=0}^\infty \left ( \frac{z}{2} \right )^n \]
		for $|z/2| < 1$, which is to say for $|z|< 2$.
		Thus we have
		\begin{align*}
			f(z) &= \frac{1}{z} \left ( \frac{1}{z} \sum_{n=0}^\infty \left ( \frac{1}{z} \right )^n \right ) \left ( \frac{-1}{2} \sum_{n=0}^\infty \left( \frac{z}{2} \right)^n \right )\\
			&= \frac{-1}{2z} \left (\frac{1}{z} + \frac{1}{z^2} + \frac{1}{z^3} + \cdots  \right) \left ( 1 + \frac{z}{2}+\frac{z^2}{4}+\frac{z^3}{8}+\cdots \right ).
		\end{align*}
		Note that the above product converges when each term converges, which is to say on the annulus $1 < |z| < 2$.
	
		Now note that the coefficient of 	$z^{-1}$ of the Laurent expansion is 
		\begin{align*}
			- \frac{1}{2} \left ( \frac{1}{2} + \frac{1}{4} +\frac{1}{8} + \cdots \right ) &=  \frac{-1}{2} \left [ \sum_{n \geq 0} (1/2)^n - 1\right ] \\
			&= -\frac{1}{2} \left (\frac{1}{1-1/2} - 1 \right)\\
			&= -\frac{1}{2}.
		\end{align*}
		
		The coefficient of $z^{0}$ is 
		\begin{align*}
			-\frac{1}{2} \left ( \frac{1}{4} + \frac{1}{8} + \frac{1}{16} +\cdots \right ) 
			&= -\frac{1}{2} \left(2 - 1 - \frac{1}{2}\right) = -1/4 \\
		\end{align*} 
		
		The coefficient of $z^1$ is
		\begin{align*}
			-\frac{1}{2} \left ( \frac{1}{8} + \frac{1}{16} + \frac{1}{32} + \cdots \right ) &= - \frac{1}{2} \left ( 2 - 1 - \frac{1}{2} - \frac{1}{4} \right )\\
			&= - \frac{1}{8}.
		\end{align*}
		 
		Therefore \[ f(z) = \cdots - \frac{1}{2z} - \frac{1}{4} - \frac{z}{8} + \cdots \]
	\end{proof}
	
Note that there is also a Laurent series which converges for the annulus $0 < |z| < 1$. This can be found by using the geometric series expansion \[\frac{1}{z-1} = \frac{-1}{1-z} = - \sum_{n=0}^\infty z^n\] which of course converges for $|z|< 1$, and using the same expansion of $\frac{1}{z-2}$ as above. The expansion which converges on the punctured unit disk is the one provided by SageMath. For another example of this, see \href{https://math.stackexchange.com/questions/2553132/laurent-series-for-different-domains}{this math StackExchange post}.

\item Let $f$ be an entire function so that $\Re f(z)$ is \emph{bounded}. Show $f$ is constant.

\begin{proof}
Let $f$ be entire so it admits a power series representation 
\[ f(z) = \sum_{n \geq 0 } \alpha_n z^n.\]
Let $\gamma_R$ be a circle of radius $R$ centered at the origin.
The Cauchy inequality tells us that
\begin{align*}
|\alpha_n| &= \frac{|f^{(n)}(0)|}{n!}\\
&\leq \frac{\max_{z \in \gamma_R}| f|}{R^n}.
\end{align*} 
Now note that for any $z$ we have $|f(z)| \leq \Re f(z) + \Im f(z)$ (the triangle inequality).
This implies that, letting $B$ be an explicit bound so that $|\Re f(z)| < B$ for all $z$.
\begin{align*}
\frac{\max_{z \in \gamma_R} |f|}{R^n} &\leq \frac{\max_{z \in \gamma_R} (\Re f(z) + \Im f(z))}{R^n}\\
& \leq \frac{ \max_{z \in \gamma_R} (B + \Im f(z))}{R^n}\\
& = \frac{B + \max_{z \in \gamma_R} \Im f(z)}{R^n}\\
& = \frac{B + R}{R^n}.\\
\end{align*}
Note that this holds for any $R$. So for any $n > 1$, we have
\[ |\alpha_n| \leq \lim_{R \rightarrow \infty} \frac{B + R }{R^n} = 0.\]
This means that the power series representation for $f$ can be simplified to
\[ f(z) = \alpha_0 + \alpha_1 z.\]
But linear functions are unbounded, and so we know $\alpha_1$ must be zero.
Hence $f(z) = \alpha_0$ is a constant function.
\end{proof}

\item Evaluate $\displaystyle \int_0^\infty \frac{\cos tx}{1+x^2} dx $ for $t \in \mathbb{R}$.

\begin{proof}
Let $R > 1$ be a real number and let $\gamma_R$ be the curve in the upper half plane 
defined by \[[-R,R] \cup \{ z \in \mathbb{C} : |z| = R, \Im( z) > 0\}.\]
Let $\gamma^\prime_R:=\{ z \in \mathbb{C} : |z| = R, \Im( z) > 0\}$.
Let \[f(z) := \frac{e^{itz}}{1+z^2}\]
note that for $z \in \mathbb{R}$, $f$ agrees with the requested integral,
since Euler's formula tells us that $e^{i t z} = \cos (tz) + i \sin t(z)$ for $z \in \mathbb{R}$.

Now note that $f$ has simple poles at $z = \pm i$. Thus, the residue theorem tells us that
\[ \int_{\gamma_R} f(z) dz = 2\pi i \res_{i} f\]
since $- i$ is not inside $\gamma_R$.
Since the pole at $i$ is simple, we can compute
\[ \res_i f = \lim_{z \rightarrow i} (z-i) f(z) = \frac{e^{i t i}}{2i} = \frac{e^{-t}}{2i}\]
so we get that 
\[ \int_{\gamma_R} f(z) dz = \pi e^{-t}.\]
Since $\gamma_R$ is the union of $[-R,0) \cup [0,R] \cup \gamma^\prime$, we may split the integral
\begin{align*}
\int_{\gamma_R} f(z) dz &= \int_{[-R,0)} f(z) dz + \int_{[0,R]} f(z) dz + \int_{\gamma^\prime_R} f(z)dz.
\end{align*}
By the estimation lemma, we know that 
\[ \left |\int_{\gamma^\prime_R} f(z) dz \right | \leq L \max_{z \in \gamma_R^\prime} |f(z)|\]
where $L = \pi R$ is the length of $\gamma^\prime_R$.
We now compute 
\begin{align*}
 \max_{z \in \gamma_R^\prime} |f(z)| &=  \max_{z \in \gamma_R^\prime} \frac{|e^{itz}|}{|1+z^2|}\\
&=  \max_{z \in \gamma_R^\prime} \frac{e^{-t\Im z}}{|1+z^2|}\\
&=  \max_{z \in \gamma_R^\prime} \frac{e^{-t\Im z}}{|z+i||z-i|}
\end{align*}
Now we collect the following inequalities:
\[ e^{-t \Im z} \leq 1 \forall z \in \gamma_R^\prime\]
and
\[ |z +i  | \geq R \; \forall z \in \gamma_R^\prime\]
which can be seen by considering $|z+i|$ as the distance between $z \in \gamma_R^\prime$
and $-i$ which is clearly at least $R$ (the distance is minimized when $z = R$).
Similarly
\[ |z - i | \geq R \; \forall z \in \gamma_R^\prime\]
which is the distance between $z$ and $+i$, minimized when $z=iR$.
Combining these inequalities, we see that 
\[\max_{z \in \gamma_R^\prime} f(z) \leq \frac{1}{R^2}.\]
Thus, we can bound the value of the integral on the upper half circle as
\[ |\int_{\gamma^\prime_R} f(z) |\leq \frac{\pi}{R}.\]
So as the radius tends to infinity, the integral vanishes.

Now, since along the real axis, $f(z) = \frac{\cos tz}{1+z^2}$ we have that it is an even function so
\[2 \int_{[0,R]} f(z) dz = \int_{[-R,0)} f(z) dz + \int_{[0,R]} f(z) dz.\]
Thus, we can rewrite the integral which we had previously broken up as
\[\pi e^{-t} = \int_{\gamma_R} f(z) dz = 2\int_{[0,R]} f(z) dz + \int_{\gamma^\prime_R} f(z)dz \leq 2\int_{[0,R]} f(z) dz + \frac\pi R.\]
Taking the limit as $R\rightarrow \infty$, we see that
\[\pi e^{-t} = 2\int_{[0,\infty]} f(z) dz + 0.\]
or that
\[ \frac{\pi e^{-t}}{2} = \int_0^\infty \frac{\cos(tx) }{1+x^2} dx \]
\end{proof}

%This can be verified in sage:
%sage: x,t = var('x,t')
%sage: f = cos(t*x)/(1+x^2)
%sage: assume(t>0)
%sage: integrate(f,x,0,oo)
%1/2*pi*e^(-t)

\setcounter{enumi}{3}

\item Determine the radius of convergence of the power series for $\frac{z}{1-e^z}$ at $z=0$.

\begin{proof}
	Let $R$ denote the radius of convergence of $f(z):= \frac{z}{1-e^z}$ at $z=0$.
	
	$R$ is the largest value such that there is a holomorphic function on $D_R = \{ z : |z| < R\}$ which agrees with of $f$ on $D_R \backslash \{0\}$
	%
	Observe that by L'H\^{o}pitals rule
	\[\lim_{z \rightarrow 0} \frac{z}{1-e^z} \stackrel{L'H}{=} \lim_{z \rightarrow 0} \frac{1}{-e^z} = -1\]
	and so $f$ is continuously extendable at $z=0$. 
	
	Further observe that $f(z)$ is holomorphic in the punctured open disk $D = \{ z : 0< |z| < 2\pi\}$. 
	
	Since we can extend $f$ to be continuous at $z=0$ by taking $f(0):=-1$, Riemann's theorem on removable singularities tells us that in fact this extension is holomorphic on $D \cup \{0\}$. Thus, since the radius of $D\cup\{0\}$ is $2\pi$, then 
	\begin{equation}\label{4.i}R \geq 2\pi .\end{equation}
	
	On the other hand, $\lim_{z \rightarrow 2\pi i} f(z) = \infty$ and so there is no \textit{holomorphic} function which agrees with $f(z)$ at $2\pi i$. Thus, \begin{equation} \label{4.ii} R\leq 2\pi. \end{equation} 
	
	Combining (\ref{4.i}) and (\ref{4.ii}) we see that $R = 2\pi$.
\end{proof}	

\item Show that there is a holomorphic function $f$ on the region $1 < |z| < 2$ such that $f(z)^2 = (z^2-1)(z^2-4)$.

%\begin{proof}
%There is a holomorphic square root in any simply connected domain not including $0$. 
%The zeros of the right hand side are $\pm 1$ and $\pm 2$.
%Thus, the domain $|z| < 2$ does not contain any zeros of $(z^2-4)$, and so there is a 
%holomorphic square root of $z^2-4$ on $|z|<2$. We now want to construct a holomorphic 
%square root of $z^2-1$ on $|z|>1$. 
%Now rewrite 
%\[ (z^2-1) = \frac{1}{1/(z^2-1)}. \]
%Note that $\frac{1}{z^2 - 1} = \frac{1}{z^2(1-\frac{1}{z^2})}$.
%Now $\frac{1}{1-\frac{1}{z^2}} = \sum_{n \geq 0} (\frac{1}{z^2})^n$ is holomorphic on $| \frac{1}{z^2} | < 1$ which is to say on $|z| > 1$. 
%Moreover $\frac{1}{z^2}$ is holomorphic on $ |z| > 0$, so the product $\frac{1}{1-z^2}$ is holomorphic on the
%intersection, $ |z|>1$.
%Composition of holomorphic functions is holomorphic, and so since $\frac{1}{z}$ is holomorphic except at $0$, then 
%there is a holomorphic expression for $(z^2-1)$ on $|z|>1$.
%\end{proof}


\item Prove that $w^2 = z^4 +1$ defines an elliptic curve.

\begin{proof} It will suffice to show that the same curve in $\mathbb{CP}^1$ has genus one.
There is a natural inclusion of $ \mathbb{C} \hookrightarrow \mathbb{CP}^1$, given by $z \mapsto \mathbb{C}\begin{pmatrix} z \\ 0 \end{pmatrix}$, and we can consider the point at infinity to be written $\begin{pmatrix}z \\ 0 \end{pmatrix}$. In this manner, we can identify points and write the elements of $\mathbb{CP}^1$ as 
$\mathbb{C} \cup \{ \infty\}$.
Now consider the covering of $\mathbb{CP}^1$ given by $(z,w) \mapsto z$ where the pair $(z,w)$ satisfies
$w^2 - z^4 -1 = 0$. This cover has degree $2$, since the highest exponent of $w$ is $2$.
There are two distinct square roots of $z^4+1$ except in a neighborhood of roots of $z^4+1$, which are
$\xi = e^{i \pi /2}, \xi^2,\xi^3,\xi^4$.


We now consider $w^2 - z^4-1$ as a polynomial in $(\mathbb{C}[z])[w]$. We now compute the order of vanishing at each 
coefficient at $\xi,\xi^2,\xi^3,\xi^4$. Since each root of $z^4+1$ has multiplicity one, the order of vanishing of the constant
coefficient $z^4+1$ is one for all powers of $\xi$. The coefficient of $w$ is $0$ and so its order of vanishing is $\infty$, and it does not depend on $\xi$. Finally, the order of vanishing of the coefficient $1 \in \mathbb{C}[z]$ of $w^2$ is $0$, which does not depend on $\xi$.

Thus, we obtain a newton polytope with vertices at $(0,0),(1,\infty), (2,1)$ for each. 
To check if there is a ramification at $\infty$, we invert coordinates, so
\begin{align*}
\frac{1}{w^2} &= \frac{1}{z^4} + 1 \\
z^4 &= w^2 + w^2z^4\\
\frac{z^4}{1+z^4} &= w^2.
\end{align*}
There are two distinct solutions, namely $\frac{\pm z^4}{1+z^4}$ so there is not a ramification at $\infty$.

Away from $\infty$ we have a slope of $\frac{1}{2}$ which tells us the ramification index 
$e_{\xi} = e_{\xi^2} = e_{\xi^3} = e_{\xi^4} = 2$.

Then the Riemann-Hurwitz formula relates the genus $g_Y$ of $Y:= \{(z,w) : w^2 = z^4+1 \}$ to the genus $g_{\mathbb{CP}^1}$ of $\mathbb{CP}^1$ (which is what is being covered) by
\[ 2g_Y -2 = n ( 2 g_{\mathbb{CP}^1} - 2) + \sum_{z \in \{\xi, \xi^2, \xi^3, \xi^4\}} (e_z - 1)\]
where $n = 2$ is the degree of the covering map and $e_z = 2$ is the index of ramification.
Thus,
\begin{align*}
2 g_Y - 2 &= 2 (2 \cdot 0 - 2) + 4 (2 -1) \\
&= -4 + 4 = 0\\
2 g_Y &= 2\\
\Rightarrow g_Y &= 1.
\end{align*}
Thus, the locus $Y$ is an elliptic curve.
\end{proof}

\item Try to define $w$ \emph{locally} as a holomorphic function of $z$, defined by the relation $w^5 - 5zw +1 = 0$. What are the branch points?

%Thoughts:
%	think of it in C[z][w] 
%     order of vanishing of coordinates is 0, oo, oo, oo, 1 @ 0, 0

\end{enumerate}


\end{document}