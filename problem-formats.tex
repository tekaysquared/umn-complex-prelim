\documentclass[10pt]{article}

\usepackage{amsfonts}
\usepackage{amsmath}
\usepackage{enumitem}
\usepackage[margin=1in]{geometry}
\usepackage{hyperref}
\usepackage{sectsty}

\DeclareMathOperator{\res}{res}

\sectionfont{\large}
\title{Complex Analysis Preliminary Written Exam Prep}
\begin{document}
\date{}
\maketitle
The goal of this document is to give a general outline of how to solve problems for the Complex Analysis prelim. See the study log for specific old prelim questions relevant to each section.

\section{Laurent Series}

Suppose we are given a function and are asked to find a Laurent series expansion (or some terms of it) centered at $0$ and convergent on some region. 
\begin{enumerate}
	\item Factor the function into pieces which can be easily rewritten as an infinite series, for example a geometric series \[\frac{1}{1- a} = \sum_{n \geq 0} a^n,\] or \[e^z = \sum_{n\geq 0} \frac{z^n}{n!}.\] 
	Historically, $\sin$ and $\cos$ have not been included, but I'll record them here just in case.
	If you are given something like $\frac{1}{z-1}$ there are two things you could do. More on this to follow.
	\item Rewrite the function as a product of infinite series
	\item Check the regions on which the series converges.\footnote{Checking convergence is important because in the past different regions have been requested. For example Spring 2019 \#2 and Fall 2018 \#2 gave functions of the form $\frac{1}{z^n-1}$ asked for convergence on $|z|<1$ and $|z|>1$ respectively. The different regions of convergence changes the answer significantly. \href{https://math.stackexchange.com/questions/2553132/laurent-series-for-different-domains}{For more see StackExchange.}} For example if $f(z) = \frac{1}{z-1}$ we could rewrite as either $f = \frac{-1}{1-z}$ or $f = \frac{1}{z(1-1/z)}$. 
	The first way converges for $|z|<1$. The second way converges for $|1/z|<1$ or $|z|>1$. Also remember that region of convergence of a product of series is the intersection of their individual regions of convergence, so if $\sum_{n\geq 0} a_n z^n$ converges for $|z|>1$ and $\sum_{n \geq 0} b_n z^n$ converges for $|z|<2$, then the product
	\[ \left ( \sum_{n\geq 0} a_n z^n\right) \left ( \sum_{n \geq 0} b_n z^n\right)\]
	converges for $1 < |z| < 2$.
	
	
	
	\item If the region is the one which was requested, write the series. 
	If you are only asked for a few terms, it can be helpful to write out the terms as a sum, for example
	\[\frac{1}{z} (a_0 + a_1 z + a_2 z^2 + \cdots ) (b_0 + b_1 z + b_2 z^2 + \cdots)\]
	or 
	\[(a_0 + a_1 z + a_2 z^2 + \cdots ) (b_0 + \frac{b_1 }{z} + \frac{b_2}{z^2} + \cdots).\]
	
	In the first case, recall multiplication of power series given by \[\sum a_n z^n \sum b_n z^n = \sum (\sum_{k=0}^n a_n b_{k-n}) z^n.\] Also, don't forget to include the factor of $\frac{1}{z}$ out front. 
	
	For the second case, each coefficient will be itself an infinite sum. For this example, the coefficient of $\frac{1}{z}$ is $\sum a_n b_{n+1}$, the constant coefficient is $\sum a_n b_n$, and the coefficient of $z$ will be $\sum a_{n+1}b_n$. Historically, the coefficients have been a geometric series, and so can be actually computed using $\sum r^n = \frac{1}{1-r}$. An older example also had a copy of $e =  \sum \frac{1}{n!}$ hidden in there.
\end{enumerate}

\section{Power series}
Suppose we are given a function $f(z)$ and asked to find the radius of convergence for its power series centered at some given point $z_0$. Some good facts to know are:
\begin{enumerate}[label = \alph*)]
\item The radius of convergence is the radius of the largest disk centered at $z_0$ on which there is a holomorphic function agreeing with the given function.
\item If $\Omega \subset \mathbb{C}$ such that $\Omega$ is simply connected (homotopic to a point) and $0 \not \in \Omega$ then there is a branch of the logarithm (call it $\log_\Omega$) which is holomorphic on $\Omega$.\footnote{Theorem 6.1.i in Chapter 3 of Stein and Shakarchi's \textit{Complex Analysis}. Note that the theorem also includes the assumption that $1 \in \Omega$, but this assumption is used to prove parts ii, iii of the theorem, not part i.}
\item \textbf{Riemann's theorem on removable discontinuities} states (among other things) that if $D \subseteq \mathbb{C}$ is open and $a \in D$, then if $f$ is a function which is holomorphic on $D-\{a\}$, then $f$ is continuously extendable if and only if it is holomorphically extendable. There are also other conclusions of the theorem, but the two above conclusions are the ones which are relevant to this type of problem.
\end{enumerate}

The general idea is to first bound the size of the disk from above, then from below. One way to bound from below is to explicitly construct a holomorphic function on a disk of the desired size which agrees with given function, by using (b) and (c) above.
One way to bound from above is to find a pole or essential singularity, and compute the distance between a pole or essential singularity and the given point $z_0$.

%\section{Complex $n$th roots}

\section{Conformal mapping}

The important theorem for a map to be conformal is that it $f$ conformal when $f^\prime$ is nonzero. For a great lecture about this, see the Herb Gross video \href{https://www.youtube.com/watch?v=s1DFa1dCss0}{Complex Variables: Lec 3. Conformal Mappings}. Professor Gross goes into great depth about the intuition of why angles are preserved. It is a bit hand-wavy but very digestible. 

A really useful conformal map is the so-called \href{http://www-users.math.umn.edu/~garrett/m/complex/notes_2014-15/07_conformal_mapping.pdf}{``Cayley map"} which is a conformal mapping between the open upper half plane $H$ and the unit disk $D$. There are many maps which go by the name ``Cayley map," but we will use the one given on \href{https://mathworld.wolfram.com/CayleyTransform.html}{Wolfram MathWorld}.

A nice way to remember the map was shown to me by Libby Farrell. 
We can think of the upper half plane as all points which are closer to $i$ than to $-i$. 
In other words, the upper half plane satisfies $| z - i | < |z- (-i)|$, which would imply $\frac{|z-i|}{|z+i|}  = \left | \frac{z-i}{z+i} \right |< 1$. 
This reminds us that $f(z) := \frac{z-i}{z+i}$ takes the upper half plane conformally to the unit disk. 
Then we can remember that the formula for the inverse of a linear fractional transform $\frac{az + b}{cz+d}$ looks like the way to take the inverse of a $2\times 2$ matrix namely $\frac{dz - b}{-cz + a}$.\footnote{This is not a coincidence! See PG's notes for further detail.} 
Thus, the inverse of $f(z) = \frac{z-i}{z+i}$ is $f^{-1}(z) = \frac{iz + i}{-z+1}$, which would thus be a conformal mapping from the disk to the upper half plane. We could check that (for example) $\frac{0+i}{-0+1}$ takes $0 \mapsto i$, and reminding us that the interior of the disk maps to the interior of the upper half plane.

There are some useful facts about the Cayley map:
\begin{enumerate}
	\item In general, a certain half-disk will be in correspondence with quadrant under some version of the Cayley map.
	\item For our $f^{-1}$, the upper-half disk will be taken to the second quadrant. If you forget, just plug in $x+iy$ and compute $f^{-1}(x+iy)$, and conditions on $x,y$ will tell you what goes where.
\end{enumerate}

Finally, remember that multiplication by a constant is a conformal mapping. Thus, since multiplication by $i$ corresponds to a 90-degree rotation of the complex plane, it is useful to remember you can shift things using that mulitplication. Finally, the map $z \mapsto z^2$ is a conformal mapping (as long as $0$ isn't in the domain of the function) which doubles angles so, for example, the upper half-unit-disk gets mapped to the slit unit-disk under this map.

\section{Show a function is constant}

There are a few different tools to show an entire function is constant based on the information you are given.

\begin{enumerate}
\item We know that an entire function $f(z) = f(x,y) = u(x,y) + i v(x,y)$ satisfies the \textbf{Cauchy-Riemann equations},
\[ \frac{\partial u}{\partial x} = \frac{\partial v}{\partial y} \; \; \; \; \frac{\partial u}{\partial y} = - \frac{\partial v}{\partial x} .\]
Thus, if you can show some of these are $0$, you know the other one is zero.

\item \textbf{Liouville's theorem} states that every bounded entire function is constant.
\item The \textbf{open mapping theorem} states that non-constant holomorphic functions are open maps (i.e. they send open sets to open sets). If you can show that the image under a holomorphic function of an open set is closed, then the function must be constant.
\end{enumerate}


\section{Compute an integral}
A classic question is to compute a real integral by passing to complex integrals first.
There are a few different ways to compute such an integral.

\begin{enumerate}
	\item The \textbf{residue theorem} states that if $f$ is a holomorphic in an open set containing a circle $C$ and its interior, except for a pole at $z_0$ in $C$ then 
	\[ \int_C f(z) dz = 2\pi i \res_{z_0} f.\]
	%
	In the case that there is a simple pole at $z_0$ we can compute
	\[ \res_{z_0} f = \lim_{z \rightarrow z_0} (z-z_0) f(z).\]
	In the case that there is a pole of order $n$ at $z_0$, we can compute
	\[ \res_{z_0} f = \lim_{z \rightarrow z_0} \frac{1}{(n-1)!} \left ( \frac{d}{dz}\right )^{n-1} (z-z_0)^n f(z).\]
	You can remember the second formula from the first by considering \[(z-z_0)^n f(z) = a_{-n} +  \cdots + a_{-1} (z-z_0)^{n-1} + G(z) (z-z_0)^n,\] and iterating the formal derivative for power series.
	
	\item The \textbf{estimation lemma} states that if $\gamma$ is a contour of length $L$, 
	then the integral \[ \left | \int_{\gamma} f(z) dz \right | \leq L \sup_{z \in \gamma} | f(z) |.\]
	This is especially helpful to show that an integral (or part of an integral) vanishes.
	
	\item As a general technique, compute a finite contour integral, split the contour into a few different pieces (usually one of which is a finite line segment $[-t,t] \subseteq \mathbb{R}$) then take the limit as $t \rightarrow \infty$.
	You can often start out with a closed contour and use the resuidue theorem to get a value, and show
	that part of the integral vanishes using the estimation lemma. If you set up the integral right, 
	then the part which is \emph{not} on the real line should vanish and so you are left with 
	the $2\pi i \res_{z_0} f$.
	
\end{enumerate}

\section{Compute the genus of a curve}
Let $\pi$ be a hyper elliptic curve with $y^2 = f(x)$ where $f$ is a square-free polynomial in $x$. Then the Riemann-Hurwitz formula\footnote{For further detail, see \href{PG's notes}{http://www-users.math.umn.edu/~garrett/m/complex/notes\_2014-15/ERNPRHH.pdf}} tells us that the genus $g_Y$ is 
\[g_Y  = \begin{cases} \frac{\deg(f)-1}{2} & \text{if } \deg(f) \text{ odd} \\ \frac{\deg{f}-2}{2} & \text{if } \deg(f) \text{ even}  \end{cases}. \]
To solve these problems rewrite the given function as $y^2 = $(a polynomial) and then use the formula.

\end{document}