\documentclass{article}

\usepackage{amsthm}
\usepackage{amsmath}
\usepackage[margin=1in]{geometry}
\usepackage{hyperref}

\title{\href{https://math.umn.edu/sites/math.umn.edu/files/exams/complexs18.pdf}{Spring 2018 Complex Analysis Preliminary Exam}}
\author{University of Minnesota}
\date{}
\begin{document}
\maketitle

Where possible, computations have been also done using SageMath code available on GitHub at \\ github.com/tekaysquared/prelims (feel free to make pull requests!)

\begin{enumerate}
	
	\item Write the first three terms of the Laurent expansion of $\displaystyle f(z) = \frac{1}{z(z-1)(z-2)}$ centered at $0$ and convergent in $|1| < z < |2|$
	
	\begin{proof}
		The core idea of the computation is to split the function into a product of power series.
		First, we observe that 
		\[ \frac{1}{z-1} = \frac{1}{z(1-1/z)}\] 
		and see the geometric series 
		\[ \frac{1}{1-1/z} = \sum_{n=0}^\infty \left( \frac{1}{z} \right)^n,\]
		which converges for $|1/z| < 1$, or equivalently $|z|>1$.
		Similarly we see that 
		\[ \frac{1}{z-2} = \frac{-1}{2(1-z/2)} = -\frac{1}{2} \sum_{n=0}^\infty \left ( \frac{z}{2} \right )^n \]
		for $|z/2| < 1$, which is to say for $|z|< 2$.
		Thus we have
		\begin{align*}
			f(z) &= \frac{1}{z} \left ( \frac{1}{z} \sum_{n=0}^\infty \left ( \frac{1}{z} \right )^n \right ) \left ( \frac{-1}{2} \sum_{n=0}^\infty \left( \frac{z}{2} \right)^n \right )\\
			&= \frac{-1}{2z} \left (\frac{1}{z} + \frac{1}{z^2} + \frac{1}{z^3} + \cdots  \right) \left ( 1 + \frac{z}{2}+\frac{z^2}{4}+\frac{z^3}{8}+\cdots \right ).
		\end{align*}
		Note that the above product converges when each term converges, which is to say on the annulus $1 < |z| < 2$.
	
		Now note that the coefficient of 	$z^{-1}$ of the Laurent expansion is 
		\begin{align*}
			- \frac{1}{2} \left ( \frac{1}{2} + \frac{1}{4} +\frac{1}{8} + \cdots \right ) &=  \frac{-1}{2} \left [ \sum_{n \geq 0} (1/2)^n - 1\right ] \\
			&= -\frac{1}{2} \left (\frac{1}{1-1/2} - 1 \right)\\
			&= -\frac{1}{2}.
		\end{align*}
		
		The coefficient of $z^{0}$ is 
		\begin{align*}
			-\frac{1}{2} \left ( \frac{1}{4} + \frac{1}{8} + \frac{1}{16} +\cdots \right ) 
			&= -\frac{1}{2} \left(2 - 1 - \frac{1}{2}\right) = -1/4 \\
		\end{align*} 
		
		The coefficient of $z^1$ is
		\begin{align*}
			-\frac{1}{2} \left ( \frac{1}{8} + \frac{1}{16} + \frac{1}{32} + \cdots \right ) &= - \frac{1}{2} \left ( 2 - 1 - \frac{1}{2} - \frac{1}{4} \right )\\
			&= - \frac{1}{8}.
		\end{align*}
		 
		Therefore \[ f(z) = \cdots - \frac{1}{2z} - \frac{1}{4} - \frac{z}{8} + \cdots \]
	\end{proof}
	
Note that there is also a Laurent series which converges for the annulus $0 < |z| < 1$. This can be found by using the geometric series expansion \[\frac{1}{z-1} = \frac{-1}{1-z} = - \sum_{n=0}^\infty z^n\] which of course converges for $|z|< 1$, and using the same expansion of $\frac{1}{z-2}$ as above. The expansion which converges on the punctured unit disk is the one provided by SageMath. For another example of this, see \href{https://math.stackexchange.com/questions/2553132/laurent-series-for-different-domains}{this math StackExchange post}.

\setcounter{enumi}{3}

\item Determine the radius of convergence of the power series for $\frac{z}{1-e^z}$ at $z=0$.

\begin{proof}
	Let $R$ denote the radius of convergence of $f(z):= \frac{z}{1-e^z}$ at $z=0$.
	
	$R$ is the largest value such that there is a holomorphic function on $D_R = \{ z : |z| < R\}$ which agrees with of $f$ on $D_R \backslash \{0\}$
	%
	Observe that by L'H\^{o}pitals rule
	\[\lim_{z \rightarrow 0} \frac{z}{1-e^z} \stackrel{L'H}{=} \lim_{z \rightarrow 0} \frac{1}{-e^z} = -1\]
	and so $f$ is continuously extendable at $z=0$. 
	
	Further observe that $f(z)$ is holomorphic in the punctured open disk $D = \{ z : 0< |z| < 2\pi\}$. 
	
	Since we can extend $f$ to be continuous at $z=0$ by taking $f(0):=-1$, Riemann's theorem on removable singularities tells us that in fact this extension is holomorphic on $D \cup \{0\}$. Thus, since the radius of $D\cup\{0\}$ is $2\pi$, then 
	\begin{equation}\label{4.i}R \geq 2\pi .\end{equation}
	
	On the other hand, $\lim_{z \rightarrow 2\pi i} f(z) = \infty$ and so there is no \textit{holomorphic} function which agrees with $f(z)$ at $2\pi i$. Thus, \begin{equation} \label{4.ii} R\leq 2\pi. \end{equation} 
	
	Combining (\ref{4.i}) and (\ref{4.ii}) we see that $R = 2\pi$.
\end{proof}	

\end{enumerate}


\end{document}