\documentclass{article}

\usepackage{amsthm}
\usepackage{amsmath}
\usepackage{amsfonts}
\usepackage[margin=1in]{geometry}
\usepackage{hyperref}

\DeclareMathOperator{\im}{im}

\title{\href{https://math.umn.edu/sites/math.umn.edu/files/exams/complexf19.pdf}{Fall 2019 Complex Analysis Preliminary Exam}}
\author{University of Minnesota}
\date{}
\begin{document}
\maketitle

Where possible, computations have been also done using SageMath code available on GitHub at \\ github.com/tekaysquared/prelims (feel free to make pull requests!)

\begin{enumerate}
	\item Give a conformal mapping from the (open) upper half-disk $H = \{z : |z|<1 \text{ and } \im(z) > 0\}$ to the slit disk
	\[D = \{ z \in \mathbb{C} : |z|< 1,\;z \not \in [0,1] \]
	
	\begin{proof}
		First, let $f(z) = 1/z$. Then $f(H) = \{ z : |z|>1 \text{ and } \im(z) > 0 \}$. Since both $f$ and $f^\prime(z) = -z^{-2}$ are never $0$, 
		$f$ is a conformal mapping.
		%
		Now let $g(z) = z^2$. Then $g(f(H)) = \{ re^{i \theta} : r > 1 \text{ and } \theta \in (0, 2\pi) \}$. 
		$g=0$ and $g^\prime = 0$ only at $z = 0$, which is not in $f(H)$ so $g \circ f$ is a conformal mapping from $H$ to $g(f(H))$.
		%
		Finally, note that $f \circ g \circ f(H) = D$. 
		Again $f^\prime$ is never zero so the composition $f \circ g \circ f : H \rightarrow D$ is a conformal mapping.
	\end{proof}

	\setcounter{enumi}{1}
	
	\item Write the first three terms of the Laurent expansion of $\displaystyle f(z) = \frac{1}{z(z-1)(z-2)}$ centered at $0$ and convergent in $|1| < z < |2|$
	
	\begin{proof}
		The core idea of the computation is to split the function into a product of power series.
		First, we observe that 
		\[ \frac{1}{z-1} = \frac{1}{z(1-1/z)}\] 
		and see the geometric series 
		\[ \frac{1}{1-1/z} = \sum_{n=0}^\infty \left( \frac{1}{z} \right)^n,\]
		which converges for $|1/z| < 1$, or equivalently $|z|>1$.
		Similarly we see that 
		\[ \frac{1}{z-2} = \frac{-1}{2(1-z/2)} = -\frac{1}{2} \sum_{n=0}^\infty \left ( \frac{z}{2} \right )^n \]
		for $|z/2| < 1$, which is to say for $|z|< 2$.
		Thus we have
		\begin{align*}
			f(z) &= \frac{1}{z} \left ( \frac{1}{z} \sum_{n=0}^\infty \left ( \frac{1}{z} \right )^n \right ) \left ( \frac{-1}{2} \sum_{n=0}^\infty \left( \frac{z}{2} \right)^n \right )\\
			&= \frac{-1}{2z} \left (\frac{1}{z} + \frac{1}{z^2} + \frac{1}{z^3} + \cdots  \right) \left ( 1 + \frac{z}{2}+\frac{z^2}{4}+\frac{z^3}{8}+\cdots \right ).
		\end{align*}
		Note that the above product converges when each term converges, which is to say on the annulus $1 < |z| < 2$.
	
		Now note that the coefficient of 	$z^{-1}$ of the Laurent expansion is 
		\begin{align*}
			- \frac{1}{2} \left ( \frac{1}{2} + \frac{1}{4} +\frac{1}{8} + \cdots \right ) &=  \frac{-1}{2} \left [ \sum_{n \geq 0} (1/2)^n - 1\right ] \\
			&= -\frac{1}{2} \left (\frac{1}{1-1/2} - 1 \right)\\
			&= -\frac{1}{2}.
		\end{align*}
		
		The coefficient of $z^{0}$ is 
		\begin{align*}
			-\frac{1}{2} \left ( \frac{1}{4} + \frac{1}{8} + \frac{1}{16} +\cdots \right ) 
			&= -\frac{1}{2} \left(2 - 1 - \frac{1}{2}\right) = -1/4 \\
		\end{align*} 
		
		The coefficient of $z^1$ is
		\begin{align*}
			-\frac{1}{2} \left ( \frac{1}{8} + \frac{1}{16} + \frac{1}{32} + \cdots \right ) &= - \frac{1}{2} \left ( 2 - 1 - \frac{1}{2} - \frac{1}{4} \right )\\
			&= - \frac{1}{8}.
		\end{align*}
		 
		Therefore \[ f(z) = \cdots - \frac{1}{2z} - \frac{1}{4} - \frac{z}{8} + \cdots \]
	\end{proof}
	
Note that there is also a Laurent series which converges for the annulus $0 < |z| < 1$. This can be found by using the geometric series expansion \[\frac{1}{z-1} = \frac{-1}{1-z} = - \sum_{n=0}^\infty z^n\] which of course converges for $|z|< 1$, and using the same expansion of $\frac{1}{z-2}$ as above. This is the one provided by SageMath. For another example of this, see \href{https://math.stackexchange.com/questions/2553132/laurent-series-for-different-domains}{this math StackExchange post}.

\setcounter{enumi}{4}
	\item Determine the radius of convergence of the power series for $z \log z$ at $z_0 = -3 + 4i$.
	
	\begin{proof}
		We will look for the largest $R$ for which there is a disk $D_R$ of radius $R$ centered at $z_0$ on which there is a holomorphic function agreeing with $z \log z$.
		%
		The product of holomorphic functions is holomorphic, so because $g(z) = z$ is entire, the radius of convergence of $z \log z$ is limited by $f(z) = \log z$. 
		
		To find the radius of convergence of $f$ at $z_0$, observe that there is no number $w \in \mathbb{C}$ such that $e^w = 0$, and so the $R$ is bounded above by $|-3 + 4i - 0| = 5$.		
		
		On the other hand recall that it is a theorem\footnote{Theorem 6.1 in Chapter 3 of Stein and Shakarchi's \textit{Complex Analysis}} that if $D$ is a simply connected region which does not contain $0$, then there is a branch of the logarithm (call it $\log_D$) which is holomorphic on $D$. Consider the (open) disk $D_5$ of radius 5 centered at $z_0$. Clearly this does not contain $0$, and so there is a holomorphic $\log_{D_5}$. Thus, we see $R \geq 5$. 
		
		Since $R \leq 5$ and $R \geq 5$, we have $R = 5$.
	\end{proof}

\end{enumerate}


\end{document}